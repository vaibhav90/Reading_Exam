\newpage
\section{Hiding Stars with Fireworks:
Location Privacy through Camouflage \cite{conf/europar/PetzoldBTU06}} \label{lect4}

\subsection{Summary} \label{lect4-sum}

\subsubsection{Context}

There has been a rapid proliferation of Location Based Services (LBS) in recent
years due to ubiquitous wireless connectivity and GPS modules integrated with 
smart phones. These LBS rely on accurate, continuous and realtime streaming of 
location data. However, revealing this information to service providers poses a 
significant privacy risk. In this paper, the authors device a method to preserve
user privacy without trading on the services offered by the LBS. 

\subsubsection{Problem}

Existing research on user privacy protection in LBS takes the approach of 
obscuring user's path, compromising the accuracy of services offered by the 
LBS. Hiding parts of user's paths can lead to degrading the the spatial accuracy, 
increased delay in reporting user location or temporarily preventing the user from
reporting locations completely. This leads to user data being less useful after
enabling privacy protection. As a result a framework is needed which can protect
the user against location tracking by the service providers at the same time
offer high quality services. 
  
\subsubsection{Contributions}




\paragraph{Results}



\paragraph{Approach}

In order to clearly understand the Bayesian models considered in this paper, a brief description of each of them is given below. Bayesian networks are Directed Acyclic Graphs (DAGs), which represent the dependencies between nodes and provide a compact representation of full joint probability distributions. Nodes represent variables, or in other words, occurrences of an event or features of an object. NBN is the simplest Bayesian network where there is only one parent node (root node) of all other nodes (child nodes of the root node). TAN is a NBN with also directional links between child nodes. In a GBN, the parent node can also be a child of some child nodes. Contrary to the three others, DBN takes the time into account, more specifically it contains a sequence of static Bayesian networks where each of them represents the state of a variable at different times.
For the evaluation, a set of contextual data from 336 undergraduate students has been aggregated and used. Students had to record their daily routines over a period of two days on campus. For the GBN, TAN and NBN model induction, Weka machine learning has been used to create three location prediction models. During this automatic learning step, similar to a process of discovery knowledge, different variables have been highlighted: the user's previous action, the user's current action, the user's location and route. Then, in order to induce the new Bayesian approach of the paper (the DBN model) four steps have been applied:

\begin{enumerate}

\item Identify domain variables;
\item Examine dependencies between domain variables and how they change over time;
\item Describe how the conditional probability distributions are constructed from the user's action and location data;
\item Develop procedurally the belief update in order to use it for propagating beliefs through the DBN.

\end{enumerate}

\subsection{Discussion} \label{lect4-disc}

This paper provides a good overview of the different Bayesian network methods for context prediction: from the simplest to the more sophisticated. However this work has several limits and/or might add some potential improvements or extensions.

\begin{itemize}

\item Since students have different habits and perhaps follow different class schedules, the results of this paper may not be accurate because recorded data may not be homogeneous. The authors could use students data of a same class in order to see the differences with the current results.

\item The paper uses a cross-validation. The other possible evaluation approach would have been to use an application for the students who had participated to the research in order to evaluate the location prediction models created. This application would have notified the students with a message containing their next possible location and they would have had to answer if the notice is correct or not.

\item Finally, the last important limit is related to the previous and concerns how we can really integrate these location prediction models in a real application. In addition, with this implementation, we should be able to see how react the models if changes occur in the students' life.

\end{itemize}