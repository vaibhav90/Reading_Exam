\newpage
\section{Hiding Stars with Fireworks:
Location Privacy through Camouflage~\cite{Meyerowitz:2009:HSF:1614320.1614358}}\label{lect4}

\subsubsection*{Context}

There has been a rapid proliferation of Location Based Services (LBS) in recent
years due to ubiquitous wireless connectivity and GPS modules integrated with 
smart phones. These LBS rely on accurate, continuous and realtime streaming of 
location data. However, revealing this information to service providers poses a 
significant privacy risk. In this paper, the authors devise a method to preserve
user privacy without trading off the quality of services offered by the LBS. 

\subsubsection*{Problem}

Existing research on user privacy protection in LBS takes the approach of 
obscuring user's path, compromising the accuracy of services offered by the 
LBS. Hiding parts of user's paths can lead to degrading the the spatial accuracy, 
increased delay in reporting user location or temporarily preventing the user from
reporting locations completely. This leads to user data being less useful after
enabling privacy protection. As a result a framework is needed which can protect
the user against location tracking by the service providers at the same time
offering high quality services. 
  
\subsubsection*{Contributions}

\begin{enumerate}
	\item The paper proposes CacheCloak, which acts as an intermediary server 
	between the users' and the LBS. The framework utilises mobility prediction in
	order to camouflage the user by requesting information for a series of 
	predicted interweaving paths instead of a single GPS coordinate. 
	
	\item CacheCloak extends "Path Confusion" technique and fixes its trivial 
	flaw which is the inherent delay caused while answering queries. Path 
	prediction and prospective caching retains the benefits of path confusion 
	without incurring the delay of path confusion. 
	
	\item The iterated Markov prediction model is robust to high number of 
	mis-predictions as the user will always see only up-to-date cached data. 
	Requesting new data for mis-predictions comes at a low cost compared to 
	necessity of privacy. 
	
	\item The authors, evaluate the proposed framework considering a realistic
	attacker model. The diffusive method, models an attacker trying to follow
	every possible way the user might go considering different speeds and 
	directions. 
	
	\item The paper proposes a quantitative measure of privacy, based on 
	the attackers ability or inability to track the user over time. 
	
	\item Finally the authors discuss a practical implementation of a distributed
	form of CacheCloak under the assumption that the intermediary server is 
	untrusted. 
	
\end{enumerate}

\subsubsection*{Approach}

\begin{enumerate}

	\item The authors carry out a trace-based simulation in order to have realistic operating
	 conditions. A city map was loaded into a simulator with virtual drivers following physical laws
	 and defined speed limits, with random placement of vehicles on the map as a bootstrapping
	 criteria. The user location was written in a file system as the simulation progressed. These
	 traces were loaded into CacheCloak chronologically, simulating a real time stream.
	 
	 \item Two cases can arise while operating CacheCloak, the submitted coordinates can already
	 have the information associated with them cached which is termed as cache hit. On the contrary, 
	 if the information for a coordinate is not cached, a path prediction needs to be performed. The
	 predicted path is extrapolated until it is connected on both ends to other predicted paths 
	 present in the cache. 
	 
	 \item Next, the entire generated path is sent to the LBS and all responses for all locations
	 are retrieved and cached. If the user deviates from the predicted path, new requests to the 
	 LBS will be triggered and corresponding results will be cached. 
	 
	 \item Based on a formulated attacker model, the authors evaluate and quantify the location 
	 privacy based on the entropy of user movement which provides a measure of attacker's uncertainty. 
	 
	 \item The results show that CacheCloak can provide users with multiple bits of entropy within 
	 a maximum of 10 minutes, which can be achieved even in sparse populations where the space time 
	 intersections are rare for two users paths to intersect to different times.  
		
\end{enumerate}

\subsubsection*{Discussion} 

\begin{enumerate}

	\item In addition to a centralised implementation of CacheCloak, the paper also discusses a distributed
	version, in which the devices will have to perform their own mobility predictions. It would
	be interesting to investigate the feasibility of the diffusion schemes and the iterative Markov model and
	quantify the cost of incorrect predictions. As incorrect predictions only lead to communication costs
	while running it on the CacheCloak Server, now it would also result in computational costs. 
	
	\item In the paper, the authors primarily investigate the application of CacheCloak to 
	vehicular mobility, more specifically, cars. However, a considerable proportion of the 
	population use mixed modes of transportation. Their mobility behaviour involves bus/train/walking	
	or all of them, in which case the predictions are not so straightforward. As a result 
	some alterations in the proposed framework needs to be made to have a practical viability. 
	
\end{enumerate}
