\newpage
\section{Limits of Predictability in Human Mobility~\cite{Song1018}} \label{lect1}

\subsection*{Summary}

\subsubsection*{Context}

Human behaviour is characterised by spontaneity, randomness and change. This paper puts
forth an important contribution in stating to what extent human mobility is predictable. 
The authors identify a metric, "Entropy" as a means to measure the potential predictability
in user mobility. Although, the paper might seem trivial, it can be seen as a base research
in order to make explicit predictions on user whereabouts. 

\subsubsection*{Problem}

The authors take an interesting approach to model human activity which is not stochastic 
based. There exist several probabilistic based models, as a result it is necessary to 
know the degree of randomness in human behaviour and the extent to which human mobility 
is predictable. Furthermore, it is important to know the bounds of predictability 
characterised by the mobility entropy, travel distance, frequency of visits, time spent 
at a particular location and heterogeneity of the visitation patterns. This papers fills this
void by experimentally analysing the above bounds and concluding that human mobility is 
characterised by high regularity and this is predictable.          

\subsection*{Contributions}

The paper dismisses many of the common assumptions associated with human mobility prediction
by experimental evaluation of several mobility datasets. Here lies the major contribution of 
the paper as it established bounds for various aspects of prediction. 

\begin{itemize}

	\item The article presents a technique to measure the entropy associated with user movement. 
	The entropies can be classified in to random entropy (number of distinct locations visited by the user), 
	temporal-uncorrelated entropy (characterises the heterogeneity of visitation patterns) and the actual
	entropy (accounts the order in which the nodes are visited). This distinction aids to arrive at the 
	conclusion that user movement can be predicted irrespective of the entropy thus dismissing the 
	general assumption that only lower entropy implies higher predictability. 
		
	\item The authors evaluate the Fano's inequality bound on predictability when a user with a given entropy 
	moves between N locations. It was discovered that, despite of the apparent randomness of the individual
	trajectories there exists a high degree of potential predictability in user movement. 
	
	\item The analysis also led to a conclusion that, the predictability across a large user base is insignificant 
	and varies from person to person. Further, it was also found that, users covering larger distances on regular 
	basis are just as predictable as users commuting in a small area. 
	
	\item Similar results were obtained when experiments were performed on diverse demographics of varying ages and 
	genders, i.e. only insignificant variations were found in regularity. This concludes that regularity and thus 
	predictability is not imposed by demographic factors, but instead by intrinsic human activities. 
	
	\item The combination of the empirically determined user entropy by the authors and Fano's inequality leads 
	to a potential 93\% average predictability in user mobility. 
	
\end{itemize}

\subsubsection*{Approach}

A dataset representing the call patterns of 10~million mobile phone users was used containing the routing tower 
location. The data was filtered to have only the users with a sufficient calling frequency and high 
movement which was further characterised individual call/motion activity. This data was processed to construct 
a time series for each user to determine their entropy, movement regularity and dependence on the demographic
and population density. 
 
\subsection*{Discussion}
\begin{itemize}


\item The used dataset was collected in a high income country where having a 93\% potential predictability
in user mobility is acceptable despite very large differences in travel distances due to the regularity 
and availability of transportation modes. However, it will be interesting to characterise similar parameters 
in the low income and densely populated countries which face much more extreme conditions
so as to generalise the findings.  

\item Although the authors calculated the bounds on predictability by defining entropy and movement regularity, 
the paper did not show how close to the maximum potential predictability, the accuracy of actual algorithms can 
come in practice. 
\end{itemize}