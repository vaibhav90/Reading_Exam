\newpage
\section{Discovering SpatioTemporal Mobility Profiles of Cellphone Users~\cite{DBLP:conf/wowmom/BayirDE09}} \label{lect3}

\subsection*{Summary}

\subsubsection*{Context}

In order to develop applications, related to context-based search and advertising, inherent
in Location Based Services, it is trivial to understand the mobility patterns and profiles of
users. These patterns need to be extracted from the raw GPS traces which are available from the 
cellphones. Different users have distinct behaviours which influence their mobility patterns, as 
a result it is crucial to discover these factors which affect the mobility path information
retrieval. This paper focuses on discovering the spatiotemporal mobility patterns and mobility
profiles from cell phone location logs.  
   

\subsubsection*{Problem}

Today, the LBS providers continuously monitor their users and log their location information, in 
the form of GPS coordinates. However, extracting meaningful information out of these raw traces is
a challenge due to the anomalies caused while tracking and user specific behaviour. On the 
contrary to the existing work which are restricted to small scale environments in order to 
study human mobility, this paper puts forth a framework "Mobility Profiler", for discovering user
mobility patterns and user profiles at a city wide level using cellular networks.  

\subsection*{Contributions}

The main contributions of the paper can be summarised as below:

\begin{itemize}
	
	\item The paper introduces formal definitions for the concepts of mobility path, mobility 
	pattern and mobility profile and the factors influencing each. 
	
	\item The authors design and implement the complete framework of Mobility Profiler to 
	discover mobility profiles from raw data. 
	
	\item The paper also presents several experiments conducted using the Reality Mining data 
	set to determine realistic thresholds for when to consider location end times, interim 
	location on a mobility path and others alike. 
	
	\item The previous studies have concluded that typical users spend 85\% of their time in
	3 to 4 locations. The paper sheds light on user behaviour and patterns during the 
	remaining 15\% of the time. 
	
	 
\end{itemize}

\subsubsection*{Approach}

The Mobility Profiler Framework consists of the following phases: 

\begin{itemize}

	\item Path Construction: Here, an ordered set of cell tower IDs corresponding to 
	user's travel path is constructed. Cell clustering is further employed to eliminate 
	oscillating cell towers and replace them with their corresponding clusters. 
	
	\item Topology Construction: The extracted travel paths of cell clusters are used to 
	construct topology of user movements. 
	
	\item Pattern Discovery: Here, the frequent mobility patterns of each user are discovered. 
	This step is carried out by employing the topology information and a string matching 
	support criteria, i.e. for every subsequent pair of cell clusters in a sequence, the 
	former one should be a neighbour of the latter one in the cell-cluster topology graph. 
	
	\item Post Processing: The extracted personal mobility patterns are then used to generate
	cellphone user profiles. 
	 
\end{itemize}


\subsubsection*{Results}
The paper presents several interesting results summarised as below:

\begin{itemize}

	\item The authors find that the average duration spent in a cell is 10~min where it is 
	defined as the duration between the cell end time and the cell start time. The average
	cell transition time was also found to be 10~min where it is defined as the time duration
	between the subsequent cell start time and the current cell end time. These values were 
	experimentally found by analysing the ratio of cell span duration and cell span transitions,
	smaller than predefined time values in the experiment phase.  
	
	\item In order to determine the right cluster in the case of oscillations, it is important
	to determine the minimum switching count. It was analytically found out that in order to 
	distinguish between oscillations due to user mobility and cell tower oscillations the 
	minimum switching threshold should be 3. 
	
	\item Further, experiments were performed to discover patterns for generating both global
	and personal frequent patterns. It was found out that frequency 	of mobility paths is inversely
	correlated with the path-length. 
	
\end{itemize}

\subsection*{Discussion}
A single cell may encompass several points of interests which are masked by the resolution 
offered by logging only the network tower ID which covers a large area. As a result, the thresholds 
presented in the paper may not hold in practicality which consists of user mobility characterised 
by short movements and stay times. It would be interesting to implement and analyse, how these
parameters change when it is applied to a dataset consisting of GPS logs mapped at a high
frequency. 
