\newpage
\section{Understanding Mobility based on GPS Data~\cite{Zheng:2008:UMB:1409635.1409677}} \label{lect2}

\subsubsection*{Context}

A considerable amount of research studies have focussed on detecting significant locations
in the user trajectory data, predicting future locations of a user or recognising user specific
activity at a particular location. On the contrary, classifying user GPS trajectories
based on transportation modes has not received substantial attention. Knowledge regarding
the transportation modes is significant in order to provide pervasive computing systems with
meaningful context information.

\subsubsection*{Problem}

To date, the transportation mode classification, relies on manual labelling by the users or
utilised GSM radio signals, which could only discriminate between simple motions such as moving
and being stationary. On the other hand, identification methods based on trivial approaches such
as velocity-based classification leads to inherent errors due to the frequency at which users
switch the travel modes. The velocity of the travel modes is also influenced by traffic conditions
and weather. In this paper, the user GPS trajectories are used to understand and
distinguish between users transportation mode, such as walking, driving or taking a bus based on
supervised learning.

\begin{figure}[h]
\centering
\includegraphics[width=0.5\textwidth]{tm1.pdf}
\caption{GPS logs and transportation mode prediction~(From~\cite{Zheng:2008:UMB:1409635.1409677})}
\end{figure}

\subsubsection*{Contributions}

The main contributions of the paper can be summarised as below:

\begin{itemize}
	\item The papers presents a change point-based segmentation technique to extract mobility 
	trajectories from raw GPS logs. 
	
	\item The technique presented in the paper to infer the mobility mode is independent 
	of any additional database of road networks or Point of Interests (POI's). Therefore it can 
	be deployed in a range of web applications such as location tagging to pictures, sharing travel routes
	and experiences in a web community. 
	
	\item The extracted features from the raw dataset are robust to traffic conditions and contain
	a significant amount of user motion. The evaluation of the proposed approach considering various
	combinations of the extracted features resulted in an accuracy of 72.8~\%. 
	
\end{itemize}


\subsubsection*{Approach}

The framework to infer the transportation mode consists of an offline learning stage and a online
processing stage. In the offline learning part, the GPS trajectories are partitioned into segments
depending on the direction change points. The individual segments are utilised to extract
several features which are used to train a classification model for the online inference stage.\newline

When the GPS trajectory is extracted, it is first partitioned into segments from which features such as
direction change rate, velocity change rate, stop rate and heading direction are computed.
Using these features, the inference model is utilised to predict the transportation mode for
a given segment in a probabilistic manner. As a classification model, decision tree is used based on
a change point based segmentation method. The change points are grouped into clusters using
a density-based clustering algorithm. The derived clusters are used to construct an undirected graph
with nodes as an individual cluster and edges being the transportation between nodes. A spatial index is
built over the graph to improve the efficiency of accessing information of each node and edge. Finally,
the probability distribution is calculated of different transportation modes on each edge.

\subsubsection*{Discussion}

The contribution of this paper can be seen as a methodology to infer the mobility mode, given raw GPS
data logs. The non-user specific model presented in the paper is independent of external sensors
such as GSM signals, heart rate and map information like road networks and bus stops. The change
point based segmentation method devised by the authors is proved to be robust to variable traffic
conditions. It partitions each GPS trajectory into separate segments of different transportation
modes while maintaining the length of one segment as long as possible. Here, the change point 
corresponds to a place where a user changes their transportation mode in the trajectory.  

\begin{itemize}

\item The technique presented, involves an offline training phase that uses the extracted features
to generate the online transportation mode inference model. It would have been interesting to quantify
the data required and computational time/complexity involved at the offline stage to arrive at a 
sufficient online prediction accuracy.

\item The paper states that, the devised approach is a non-user specific model. However, sufficient
evaluation of the experimental results regarding generalisation of user specific models is not presented
in the paper.

\item The authors derive features from the raw GPS logs to train the inference model. Although,
it is evident that a combination of certain features results in a sufficient prediction accuracy, a
combination of a particular features reduces the accuracy. The reasoning behind such a choice is
lacking in the evaluation section.

\item Considering that, the majority of the population owning a smart phone has Internet connection, 
it would be interesting to map the extracted segments to road maps and points of interests in the 
training phase in order to increase the prediction accuracy.

\item The segmentation technique, extracts segments from consecutive GPS points. However, there
exists some segments, that do not offer any valuable information for feature extraction. It would
also be interesting to extract the only segments providing valuable information based on other
attributes.
\end{itemize}