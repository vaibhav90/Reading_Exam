\newpage
\section{Motivation} \label{motivation}
In recent years we have witnessed a proliferation of mobile devices with global positioning 
(GPS) functionality and internet connectivity. This has led to a rapid emergence and a notable 
progress in the development of Location-based Service (LBS). The typical examples of LBS
include automotive traffic monitoring, network resource allocation, location based 
targeted advertisements and social networking. A large amount of user data is collected by these 
companies which goes into training user specific learning models to predict user mobility and
behavioural patterns. Although LBS offer valuable services, revealing personal location data to
potentially untrustworthy service providers raises several privacy concerns.\newline

The heart of this thesis lies in augment the current mobility prediction learning frameworks in order
to incorporate privacy awareness as a key paradigm. Meanwhile researchers today, use cryptographic
techniques and Location Privacy Preserving Mechanisms (LPPM) for privacy preservation which are 
computationally complex and does not facilitate collaborative learning. We will also explore 
distributed means of computing user independent learning models in realtime maintaining the 
utility-privacy space. According to recent surveys, 55\% of LBS users have shown concern towards 
loss of their location privacy and about 50\% of U.S. residents who have a profile on social 
networking sites are concerned about their privacy. It is clear that the success of LBS depends on 
the location privacy in the near future.\newline

Since the major component of this research lies in a privacy aware mobility prediction framework, we
limit the literature survey to these two critical topics, mobility prediction and privacy aware 
computation. Firstly, it is important to understand to what extent, human mobility is predictable. 
To this end, we review a classical paper which provides insights into human mobility and the 
prediction limits. Once the GPS data is collected, it is necessary to understand how to extract 
the mobility patterns of individuals. We perform a thorough research in this area and review two
papers which highlight how to discover mobility patterns and user profiles from raw GPS logs. Next, 
we review literature related to existing privacy preserving mobility prediction models. In this
area we present an article which depicts a technique to preserve location privacy maintaining the 
quality of services offered by the LBS. We highlight several limitations of the method explained. 
Finally, we survey machine learning techniques which can be utilised to construct user mobility 
models. In this area, we review a paper which presents an novel architecture of privacy-preserving 
distributed learning approach.  



